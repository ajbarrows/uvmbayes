Intuitively, this should be true, since the conditional probability is informed by data which would increase our confidence that there are no fish.\\
The conditional probability that there are no fish in the river given that we caught no fish can be \begin{align*}
    P(\theta=N|X=0) = \frac{P(X=0|\theta=N)\cdot P(\theta=N)}{P(X=0)}
\end{align*}
Expanding the marginal $P(X=0)$ using the law of total probability, we have all of the probabilities required to calculate the conditional probability:
\begin{align*}
    P(\theta=N|X=0) &= \frac{P(X=0|\theta=N)\cdot P(\theta=N)}{P(X=0|\theta=N)\cdot P(\theta=N)+P(X=0|\theta=F)\cdot P(\theta=F)}\\
    &= \frac{(1)(0.2)}{(1)(0.2)+(0.1)(0.8)}\\
    &= 0.714
\end{align*}
Comparing the conditional probability and the marginal, we see that our intuition was correct and the former is indeed greater than the latter:
\begin{align*}
    P(\theta=N|X=0) = 0.714 > P(\theta=N) = 0.20
\end{align*}