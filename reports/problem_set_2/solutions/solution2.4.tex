Taking another derivative with respect to u, we have:
\begin{align*}
    & \frac{d}{du}(k_1+k_2)\cdot F_\theta(\hat{u}) - k_2 = (k_1+k_2)\cdot f_\theta(\hat{u}) > 0\\
    & \text{ (since every pdf $>0$ and $k_1$ and $k_2$ are non-negative)}
\end{align*}
When the second derivative of some function equals zero, there is an inflection point since the slope changes from increasing to decreasing or vice versa. When the second derivative is less than zero, the slope is decreasing, which corresponds to concave down regions. Finally, when the second derivative is greater than zero, the slope is increasing, which corresponds to concave up regions. Whereas a maximum can only occur in a concave down region, a minimum can only occur in a concave up region. Thus, a positive second derivative is a requirement for a minimum at a particular location and it also rules out a maximum at that location.